\documentclass[twocolumn,10pt]{IFTOMM}
\newcommand{\gbf}[1] {\mbox{\boldmath${#1}$\unboldmath}}% inserts 'g'reek 'b'old 'f'ace symbols eg. \gbf{\alpha}.
\newcommand{\R}{\hbox{I \kern -.5em R}}
\usepackage{epsfig}

% Für Deutsche Umlaute
\usepackage{ngerman}
\usepackage[utf8]{inputenc}

\begin{document}
%PAPER NUMBER: will be provided by the organization for accepted paper
\def\papernumber{IK xxx}
\def\conference_name{15th IFToMM World Congress, Krakow, Poland, 30 June--4 July, 2019}
%PAPER TITLE
\title{Resolution of Functional Redundancy for 3T2R Robot Tasks  \\ using Two Sets of Reciprocal Euler Angles}

\author{
    \begin{tabular}{cccc}
    M.\ Schappler\thanks{moritz.schappler@imes.uni-hannover.de}
& S.\ Tappe\thanks{svenja.tappe@imes.uni-hannover.de}
& T.\ Ortmaier\thanks{tobias.ortmaier@imes.uni-hannover.de}\\
    Leibniz Universität & Leibniz Universität & Leibniz Universität\\
Hannover, Germany & Hannover, Germany & Hannover, Germany
    \end{tabular}
}

%ESSENTIAL, will not work otherwise
\maketitle

%ABSTRACT
\begin{abstract}
Tasks like welding or milling with three translational and only two rotational degrees of freedom (``3T2R'') are of high industrial relevance but are rather scarcely addressed in scientific publications.
Existing solutions for resolution of the functional redundancy of robotic manipulators performing these tasks either expand the full kinematic formulation or reduce it in intermediate steps.
This paper presents an approach to reduce the kinematic formulation from the start to solve the problem in a simple way.
This is done by using a different set of reciprocal Euler angles to describe the end-effector orientation and the orientation error in inverse kinematics.
\end{abstract}

%KEYWORDS
\begin{keywords}
functional redundancy, redundancy resolution, reciprocal Euler angles, inverse kinematics, robots, five-DoF task, 3T2R task
\end{keywords}

\section{Introduction}

Das sind alle Quellen: \cite{Baron2000,HuoBar2005,Huo2006,HuoBar2008,Huo2009,HuoBar2011,GoldenbergBenFen1985,Zlajpah2017,Yoshikawa1984,LegerAng2016,LegerAng2015}




\section{Acknowledgements}

The financial support from the Deutsche Forschungsgemeinschaft (German Research Foundation, DFG) under grant 341489206 are gracefully acknowledged.

% BIBLIOGRAPHY
\bibliographystyle{ieeetr}
\bibliography{ikfr_ref}

\end{document}
